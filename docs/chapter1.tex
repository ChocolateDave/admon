\chapter{INTRODUCTION}
\pagestyle{fancy}

Flight delay is a serious and growing issue for the air transportation system \citep{kafle2016modeling}. Before the global pandemic, the US' estimated annual total delay cost was around \$30 billion \citep{ball2010total}. Causes for flight delays vary from weather disruption to human factors or both; examples are terminal configuration, enroute weather, and operational errors \citep{hsiao2006econometric}. Identifying underlying patterns and building a robust prediction model are two major approaches to addressing this issue. In previous decades, statistical analysis has been a practical methodology for understanding local or periodic flight delays \citep{doi:10.2514/6.2002-5866,mazzeo2003competition,ABDELATY2007355}. But with improving sensing and big-data technology, data-driven machine learning is emerging nowadays. Decomposition and classical machine learning methods can help reduce dimensionality and measure correlations between potential key factors and flight delays \citep{8911489,gorripaty2017identifying,grabbe2014clustering}. In recent years, deep learning algorithms, well-known for their capability to tackle large, high-dimensional datasets, have also been widely adopted \citep{8903554,9391561}. 

However, flight delay is usually not a local phenomenon but rather a system issue that propagates across the air traffic network. Due to the underlying spatial-temporal connectivity, flight delays aggregated by airports can highly correlate to the others \citep{li2019spectral}. Therefore, analysis and prediction models that overlook the network structure can be myopic. Even if we account for the fact that airports are within a network, it's essential to determine the structure of the network and the connections among the airports. To address these challenges in the existing works, we propose a data-driven graph model named Adaptive Deep Modularity Network. It's capable of simultaneously learning airport connections and clustering based on the delay patterns.


